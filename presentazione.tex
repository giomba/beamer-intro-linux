%~ Questo sorgente è stato scritto da Giovan Battista Rolandi
%~ ed è rilasciato sotto licenza GPL3

% \documentclass[handout]{beamer} %% stile di stampa su carta

%
% QUANDO IO DIO CHE BISOGNA USARE GLI SPAZI E NON LE TABULAZIONI
% HO FOTTUTAMENTE RAGIONE, PERCHÈ GLI SPAZI SONO SPAZI OVUNQUE
% MENTRE LE TABULAZIONI CAMBIANO SIGNIFICATO A SECONDA DEL DISPOSITIVO
% E DELL'EDITOR CHE SI USA
% QUEL BASTARDO DI GEANY MI HA DISTRUTTO L'INDENTAZIONE DI QUESTO FILE TEX
% CHE HO RIFATTO VELOCEMENTE ALLA MENO PEGGIO. MENO MALE NON È UN PROGRAMMA
% IN PYTHON
%

\documentclass{beamer}
\usepackage[italian]{babel}
\usepackage[utf8]{inputenc}
\usepackage[font=scriptsize]{caption}
\usepackage{fancyvrb}

\usetheme{Madrid}
\colorlet{beamer@blendedblue}{green!40!black}
%~ \usetheme{Luebeck}
%~ \usetheme{Berkeley}
%~ \usetheme{Goettingen}
%~ \usetheme{Rochester}
%~ \usetheme{Singapore}
\setbeamertemplate{caption}[numbered]
\setbeamercovered{dynamic}

\beamertemplatenavigationsymbolsempty

\title{Introduzione al software libero}
%\subtitle{}
\logo{\includegraphics[width=5em]{img/logo-linuxday.pdf}}
\author{giomba}
\date{5 settembre 2017}
\institute{GOLEM Empoli}

\begin{document}

\begin{frame}
  \maketitle
  \tableofcontents
\end{frame}

\begin{frame}[fragile]
\frametitle{Un po' di storia}

    \begin{block}{Anni 1960}
	\begin{minipage}{.15\linewidth}
	    \includegraphics[width=.9\linewidth]{img/meeting.png}
	\end{minipage}
    \begin{minipage}{.8\linewidth}
    I programmatori di computer sono soliti condividere il proprio lavoro
    \begin{exampleblock}{Codice}
    \begin{minipage}{.45\linewidth}
        \begin{Verbatim}[fontsize=\scriptsize]
int main() {
    int risultato = 1,
    base = 2,
    esponente = 4;
    for (int i = 0;
    i < esponente;
    ++i) {
    risultato *= base;
    }
}
        \end{Verbatim}
    \end{minipage}
    \begin{minipage}{.45\linewidth}
        \begin{Verbatim}[fontsize=\scriptsize]
55
48 89 e5
c7 45 f0 01 00 00 00
[...]
8b 45 f4
3b 45 fc
7d 10
8b 45 f0
0f af 45 f8
89 45 f0
83 45 f4 01
eb e8
b8 00 00 00 00
5d
c3
        \end{Verbatim}
    \end{minipage}
    \end{exampleblock}
    \end{minipage}
    \end{block}
\end{frame}

\begin{frame}
\frametitle{Chiusura del software}

    \begin{block}{Anni 1970}
    \begin{itemize}
        \item
        \begin{minipage}{.2\linewidth}
            \includegraphics[width=.9\linewidth]{img/lock.png}
        \end{minipage}
        \begin{minipage}{.7\linewidth}
            Interessi commerciali impongono la nascita di "accordi di non-divulgazione"
        \end{minipage}
        \pause
        \item
        \begin{minipage}{.2\linewidth}
            \visible<2->{\includegraphics[width=.9\linewidth]{img/laserprinter.png}}
        \end{minipage}
        \begin{minipage}{.7\linewidth}
            La Xerox regala una nuova stampante laser dal software chiuso all'IA~Lab del MIT
        \end{minipage}
    \end{itemize}
    \end{block}
\end{frame}


\begin{frame}
    \frametitle{Richard Stallman, il free software e GNU}

    \begin{minipage}{.4\linewidth}
    \begin{figure}
    \includegraphics[width=.9\linewidth]{img/rms.jpeg}
    \caption{Richard~Matthew Stallman}
    \end{figure}
    \end{minipage}
    \pause
    \begin{minipage}{.55\linewidth}
    \begin{block}{Software libero}
        \begin{enumerate}
        \setcounter{enumi}{-1}
        \item libertà di poter utilizzare il programma per qualunque scopo
        \item libertà di poter studiare il funzionamento del programma
        \item libertà di poter modificare il programma
        \item libertà di poter ridistribuire il programma modificato
        \end{enumerate}
    \end{block}
    \pause
    \begin{block}{1984}
        \begin{minipage}{.2\linewidth}
        \visible<3->{\includegraphics[width=1\linewidth]{img/gnu.pdf}}
        \end{minipage}
        \begin{minipage}{.75\linewidth}
        Nasce GNU, sistema operativo completamente libero basato su Unix
        \end{minipage}
    \end{block}
    \end{minipage}
\end{frame}

\begin{frame}
    \frametitle{Linus Torvalds e Linux}

    \begin{minipage}{.4\linewidth}
    \begin{figure}
        \includegraphics[width=.9\linewidth]{img/torvalds.jpeg}
        \caption{Linus Benedict Torvalds}
        \end{figure}
    \end{minipage}
    \begin{minipage}{.55\linewidth}
        \begin{block}{Perché}
            \begin{itemize}
                \item
                    \begin{minipage}{.2\linewidth} \visible<1->{\includegraphics[width=.9\linewidth]{img/i-3-unix.pdf}} \end{minipage}
                    \begin{minipage}{.75\linewidth} All'Università si appassiona ai sistemi Unix \end{minipage}
                    \pause
                \item
                    \begin{minipage}{.2\linewidth} \visible<2->{\includegraphics[width=.9\linewidth]{img/pc-i386.png}} \end{minipage}
                    \begin{minipage}{.75\linewidth} Compra un PC i386 a rate \end{minipage}
                    \pause
                \item
                    \begin{minipage}{.2\linewidth} \visible<3->{\includegraphics[width=.9\linewidth]{img/minix.pdf}} \end{minipage}
                    \begin{minipage}{.75\linewidth} Installa Minix-Unix sul PC \end{minipage}
                    \pause
                \item
                    \begin{minipage}{.2\linewidth} \visible<4->{\includegraphics[width=.9\linewidth]{img/error.pdf}} \end{minipage}
                    \begin{minipage}{.75\linewidth} Impossibilità di modificare liberamente Minix \end{minipage}
                    \pause
            \end{itemize}
        \end{block}
    \begin{block}{1991}
        \begin{minipage}{.2\linewidth}
        \visible<5->{\includegraphics[width=1\linewidth]{img/tux.pdf}}
        \end{minipage}
        \begin{minipage}{.75\linewidth}
        Nasce il kernel Linux
        \end{minipage}
    \end{block}
    \end{minipage}
\end{frame}

%~ \begin{frame}
    %~ \frametitle{Lo sviluppo di GNU/Linux}

    %~ \begin{itemize}
        %~ \item 1984 -- Nasce il sistema operativo GNU
        %~ \item 1991 -- Nasce il kernel Linux
        %~ \item 1992 -- Il kernel Linux viene rilasciato sotto licenza GPL
        %~ \item 1993 -- Nascono Slackware e Debian
        %~ \item 1994 -- Nascono Suse e RedHat
        %~ \item 2004 -- Nasce Ubuntu
        %~ \item 2006 -- Nasce Linux Mint
    %~ \end{itemize}
%~ \end{frame}

\begin{frame}
    \frametitle{Le ragioni del successo}

    \begin{itemize}
        \item
            \begin{minipage}{.15\linewidth} \visible<1->{\includegraphics[width=.9\linewidth]{img/no-cost.pdf}} \end{minipage}
            \begin{minipage}{.8\linewidth} Costo nullo del prodotto \end{minipage}
            \pause
        \item
            \begin{minipage}{.15\linewidth} \visible<2->{\includegraphics[width=.9\linewidth]{img/servers.png}} \end{minipage}
            \begin{minipage}{.8\linewidth} Supporto multiprocessore e multipiattaforma \end{minipage}
            \pause
        \item
            \begin{minipage}{.15\linewidth} \visible<3->{\includegraphics[width=.9\linewidth]{img/apache.png}} \end{minipage}
            \begin{minipage}{.8\linewidth} Server web Apache \end{minipage}
            \pause
        \item
            \begin{minipage}{.15\linewidth} \visible<4->{\includegraphics[width=.9\linewidth]{img/redhat.png}} \end{minipage}
            \begin{minipage}{.8\linewidth} Prodotti commerciali con hardware certificato \end{minipage}
    \end{itemize}
\end{frame}

\begin{frame}
    \frametitle{Potenzialità}

    \begin{columns}
    \begin{column}{.25\textwidth}
        \includegraphics[width=.9\linewidth]{img/knoppix.png}
        \centering
        Sistemi Live
    \end{column}

    \begin{column}{.25\textwidth}
        \includegraphics[width=.9\linewidth]{img/raspberry.png}
        \centering
        Minicomputer
    \end{column}

    \begin{column}{.25\textwidth}
        \includegraphics[width=.9\linewidth]{img/android.png}
        \centering
        Smartphone
    \end{column}

    \begin{column}{.25\textwidth}
        \includegraphics[width=.9\linewidth]{img/router-decoder.png}
        \centering
        Modem, Router
    \end{column}
    \end{columns}

\end{frame}

%~ \begin{frame}
    %~ \frametitle{Distribuzioni Linux}

    %~ \begin{columns}

    %~ \begin{column}{.25\textwidth}
        %~ \includegraphics[width=.45\textwidth]{img/gnu.pdf}
        %~ \includegraphics[width=.45\textwidth]{img/tux.pdf}
        %~ \centering
        %~ GNU/Linux
    %~ \end{column}

    %~ \begin{column}{.25\textwidth}
        %~ \includegraphics[width=.8\textwidth]{img/firefox.pdf}
        %~ \centering
        %~ Browser
    %~ \end{column}

    %~ \begin{column}{.25\textwidth}
        %~ \includegraphics[width=1\textwidth]{img/libreoffice.png}
        %~ \centering
        %~ Suite Ufficio
    %~ \end{column}


    %~ \begin{column}{.25\textwidth}
        %~ \includegraphics[width=1\textwidth]{img/gnu.pdf}
        %~ \centering
        %~ GNU/Linux
    %~ \end{column}

    %~ \begin{column}{.25\textwidth}
        %~ \includegraphics[width=1\textwidth]{img/gnu.pdf}
        %~ \centering
        %~ GNU/Linux
    %~ \end{column}

    %~ \begin{column}{.25\textwidth}
        %~ \includegraphics[width=1\textwidth]{img/gnu.pdf}
        %~ \centering
        %~ GNU/Linux
    %~ \end{column}

    %~ \end{columns}

%~ \end{frame}

\begin{frame}
    \frametitle{Le Distribuzioni}

    \begin{minipage}[b][.35\textheight][t]{.3\textwidth}
    \includegraphics[width=.45\textwidth]{img/gnu.pdf}
    \includegraphics[width=.45\textwidth]{img/tux.pdf}\\
    \centering
    GNU/Linux
    \end{minipage}\hfill
    \begin{minipage}[b][.35\textheight][t]{.3\textwidth}
    \includegraphics[width=.7\textwidth]{img/firefox.pdf}\\
    \centering
    Browser
    \end{minipage}\hfill
    \begin{minipage}[b][.35\textheight][t]{.3\textwidth}
    \includegraphics[width=.7\textwidth]{img/libreoffice.png}\\
    \centering
    Suite Ufficio
    \end{minipage}\\[0.5em]

    \begin{minipage}[b][.35\textheight][t]{.3\textwidth}
    \includegraphics[width=.7\textwidth]{img/oss.png}\\
    \centering
    Utilità
    \end{minipage}\hfill
    \begin{minipage}[b][.35\textheight][t]{.3\textwidth}
    \includegraphics[width=.7\textwidth]{img/ubiquity.png}\\
    \centering
    Installer
    \end{minipage}\hfill
    \begin{minipage}[b][.35\textheight][t]{.3\textwidth}
    \includegraphics[width=.7\textwidth]{img/mint-dvd.png}\\
    \centering
    Distribuzione
    \end{minipage}
\end{frame}

\begin{frame}
    \frametitle{Le distribuzioni più famose}

    \begin{minipage}[b][.35\textheight][t]{.3\textwidth}
    \includegraphics[width=.7\textwidth]{img/logo-debian.pdf}\\
    \end{minipage}\hfill
    \begin{minipage}[b][.35\textheight][t]{.3\textwidth}
    \includegraphics[width=.7\textwidth]{img/logo-ubuntu.png}\\
    \centering
    Ubuntu
    \end{minipage}\hfill
    \begin{minipage}[b][.35\textheight][t]{.3\textwidth}
    \includegraphics[width=.7\textwidth]{img/logo-linuxmint.pdf}\\
    \centering
    LinuxMint
    \end{minipage}\\[0.5em]

    \begin{minipage}[b][.35\textheight][t]{.3\textwidth}
    \includegraphics[width=.7\textwidth]{img/logo-fedora.pdf}\\
    \centering
    Fedora
    \end{minipage}\hfill
    \begin{minipage}[b][.35\textheight][t]{.3\textwidth}
    \includegraphics[width=.7\textwidth]{img/logo-centos.pdf}\\
    \end{minipage}\hfill
    \begin{minipage}[b][.35\textheight][t]{.3\textwidth}
    \includegraphics[width=.7\textwidth]{img/redhat.png}\\
    \centering
    RHEL
    \end{minipage}

\end{frame}

\begin{frame}
    \frametitle{Tutte le distribuzioni}

    \includegraphics[width=1\linewidth]{img/linux-distribution-timeline.pdf}
    \centering
    Linux Distribution Timeline
\end{frame}

\begin{frame}
    \frametitle{Comunità}

    \begin{block}{anni 1990}
    \begin{itemize}
        \item Nascono i LUG -- Linux Users Group
        \item 1994 -- Nasce la Italian Linux Society
        \item 2000 -- Nasce il GOLEM
    \end{itemize}
    \end{block}
    \pause

    \begin{block}{GOLEM - Gruppo Operativo Linux Empoli}
        \begin{minipage}{.2\linewidth}
            \visible<2->{\includegraphics[width=1\linewidth]{img/GOLEM-logo.pdf}}
        \end{minipage}
        \begin{minipage}{.75\linewidth}
            \begin{itemize}[<+->]
                \item Ore del GOLEM
                \item Arduino Project Day
                \item Trashware
                \item Corsi, campagne di sensibilizzazione, eventi promozionali, Linux Day
            \end{itemize}
        \end{minipage}
    \end{block}
\end{frame}

\begin{frame}
  \frametitle{Introduzione al software libero}

    \begin{block}{Ore del GOLEM}
    \centering
    \begin{minipage}{.1\linewidth}
      \includegraphics[width=.9\linewidth]{img/GOLEM-logo.pdf}
    \end{minipage}
    \begin{minipage}{.8\linewidth}
    \centering
    Questa presentazione è stata preparata per\\
    GOLEM -- Gruppo Operativo Linux Empoli\\
    % in occasione del Linux Day 2015\\
    da Giovan Battista Rolandi (giomba)\\
    giomba@linux.it\\
    GPG Public ID: 5F94294D
    \end{minipage}
    %~ \begin{minipage}{.1\linewidth}
      %~ \includegraphics[width=.9\linewidth]{img/linuxday-logo.png}
    %~ \end{minipage}
    \end{block}

  \begin{block}{Licenza}
    \centering
    \begin{minipage}{.18\linewidth}
      \begin{figure}
        \centering
        \includegraphics[width=.9\linewidth]{img/gnu.pdf}
      \end{figure}
    \end{minipage}
    \hfill
    \begin{minipage}{.6\linewidth}
      \centering
      Il sorgente di questa presentazione\\
      è software libero,\\
      viene rilasciato sotto licenza GPLv3,\\
      ed è consultabile presso golem.linux.it
    \end{minipage}
    \hfill
    \begin{minipage}{.18\linewidth}
      \begin{figure}
        \centering
        \includegraphics[width=.9\linewidth]{img/gpl3.pdf}
      \end{figure}
    \end{minipage}

  \end{block}

\end{frame}

\end{document}
